\section{Solution Properties}

\begin{frame}{Optimal Solutions to (LP)}
  \begin{mytheorem}[Nemhauser-Trotter]
    Let $x$ be an extreme point of the polytope defined by the constrains of (LP) we have $x_v \in \{0, \frac{1}{2}, 1\}$ for every $v \in V$.
  \end{mytheorem}
\end{frame}

\begin{frame}[allowframebreaks]{Optimal Solutions to (LP)}
  \textit{Proof.} Let $x$ be an extreme point. Let $U\subset V$ be the set of vertices such that $x_v \notin \{0, \frac{1}{2}, 1\}$ for every $v \in U$. Suppose for contradiction that $U$ is non-empty.

  \begin{center}
    \begin{tikzpicture}
      % Draw the main line segment
      \draw[thick, -] (0,0) -- (4,0);

      % Mark the key points
      \draw[] (0,0) node[above=3pt] {$0$};
      \draw[] (2,0) node[above=3pt] {$\frac{1}{2}$};
      \draw[] (4,0) node[above=3pt] {$1$};

      % Add a variable point x_v
      \draw[fill=gustave] (0.5,0) circle (2pt) node[above=3pt] {$x_v$};

      % Add tick marks
      \draw (0,0.1) -- (0,-0.1);
      \draw (2,0.1) -- (2,-0.1);
      \draw (4,0.1) -- (4,-0.1);

      % Add curly brace from 0 to x_v
      \draw[thick,decorate,decoration={brace,amplitude=3pt,mirror}] (0,-0.2) -- (0.5,-0.2);
    \end{tikzpicture}
    \hspace{1cm}
    \begin{tikzpicture}
      % Draw the main line segment
      \draw[thick, -] (0,0) -- (4,0);

      % Mark the key points
      \draw[] (0,0) node[above=3pt] {$0$};
      \draw[] (2,0) node[above=3pt] {$\frac{1}{2}$};
      \draw[] (4,0) node[above=3pt] {$1$};

      % Add a variable point x_v
      \draw[fill=gustave] (1.5,0) circle (2pt) node[above=3pt] {$x_v$};

      % Add tick marks
      \draw (0,0.1) -- (0,-0.1);
      \draw (2,0.1) -- (2,-0.1);
      \draw (4,0.1) -- (4,-0.1);

      % Add curly brace from 0 to x_v
      \draw[thick,decorate,decoration={brace,amplitude=3pt,mirror}] (1.5,-0.2) -- (2,-0.2);
    \end{tikzpicture}

    \vspace{0.5cm}

    \begin{tikzpicture}
      % Draw the main line segment
      \draw[thick, -] (0,0) -- (4,0);

      % Mark the key points
      \draw[] (0,0) node[above=3pt] {$0$};
      \draw[] (2,0) node[above=3pt] {$\frac{1}{2}$};
      \draw[] (4,0) node[above=3pt] {$1$};

      % Add a variable point x_v
      \draw[fill=gustave] (2.5,0) circle (2pt) node[above=3pt] {$x_v$};

      % Add tick marks
      \draw (0,0.1) -- (0,-0.1);
      \draw (2,0.1) -- (2,-0.1);
      \draw (4,0.1) -- (4,-0.1);

      % Add curly brace from 0 to x_v
      \draw[thick,decorate,decoration={brace,amplitude=3pt,mirror}] (2,-0.2) -- (2.5,-0.2);
    \end{tikzpicture}
    \hspace{1cm}
    \begin{tikzpicture}
      % Draw the main line segment
      \draw[thick, -] (0,0) -- (4,0);

      % Mark the key points
      \draw[] (0,0) node[above=3pt] {$0$};
      \draw[] (2,0) node[above=3pt] {$\frac{1}{2}$};
      \draw[] (4,0) node[above=3pt] {$1$};

      % Add a variable point x_v
      \draw[fill=gustave] (3.5,0) circle (2pt) node[above=3pt] {$x_v$};

      % Add tick marks
      \draw (0,0.1) -- (0,-0.1);
      \draw (2,0.1) -- (2,-0.1);
      \draw (4,0.1) -- (4,-0.1);

      % Add curly brace from 0 to x_v
      \draw[thick,decorate,decoration={brace,amplitude=3pt,mirror}] (3.5,-0.2) -- (4,-0.2);
    \end{tikzpicture}
  \end{center}

  Take the minimum distance $\epsilon$ from $x_v$ to the closest of $\{0, \frac{1}{2}, 1\}$ for every $v \in U$ i.e.
  $$\epsilon = \min_{v \in U} \min \left\{ |x_v - 0|, \left| x_v - \frac{1}{2} \right|, |x_v - 1| \right\}.$$

  Perturb $x$ at each $v \in U$ by $ \epsilon$ by two different ways to get two new solutions $x^+$ and $x^-$ defined as follows:
  $$x^+(v) = \begin{cases}
      x_v + \epsilon & \text{if } v \in U \text{ and } x_v < \frac{1}{2}, \\
      x_v - \epsilon & \text{if } v \in U \text{ and } x_v > \frac{1}{2}, \\
      x_v            & \text{otherwise}
    \end{cases} \quad \quad
    x^-(v) = \begin{cases}
      x_v - \epsilon & \text{if } v \in U \text{ and } x_v < \frac{1}{2}, \\
      x_v + \epsilon & \text{if } v \in U \text{ and } x_v > \frac{1}{2}, \\
      x_v            & \text{otherwise}.
    \end{cases}$$
  We have $x=\dfrac{1}{2}(x^+ + x^-)$.

  \break

  Two see that both $x^+$ and $x^-$ are feasible, there are two cases

  \begin{itemize}
    \item[(i)] If an edge $uv \in E$ has $x_u < \frac{1}{2}$ and $x_v > \frac{1}{2}$, then
          $$x^+_u + x^+_v = x^-_u + x^-_v = x_u + x_v \ge 1.$$

          \begin{center}
            \begin{tikzpicture}
              % Draw the main line segment
              \draw[thick, -] (0,0) -- (8,0);

              % Mark the key points
              \draw[] (0,0) node[above=3pt] {$0$};
              \draw[] (4,0) node[above=3pt] {$\frac{1}{2}$};
              \draw[] (8,0) node[above=3pt] {$1$};

              % Add tick marks
              \draw (0,0.1) -- (0,-0.1);
              \draw (4,0.1) -- (4,-0.1);
              \draw (8,0.1) -- (8,-0.1);

              % Original points x_u and x_v
              \draw[fill=gustave] (2,0) circle (2pt) node[above=3pt] {$x_u$};
              \draw[fill=red] (6.4,0) circle (2pt) node[above=3pt] {$x_v$};

              % x^+ points (red)
              \draw[fill=gustave] (2.5,0) circle (2pt) node[above=3pt] {$x^+_u$};
              \draw[fill=red] (5.9,0) circle (2pt) node[above=3pt] {$x^+_v$};

              % x^- points (blue)
              \draw[fill=gustave] (1.5,0) circle (2pt) node[above=3pt] {$x^-_u$};
              \draw[fill=red] (6.9,0) circle (2pt) node[above=3pt] {$x^-_v$};
            \end{tikzpicture}
          \end{center}

          The constraint $x_u + x_v \ge 1$ is satisfied for both $x^+$ and $x^-$ since the sum remains at least 1.
    \item[(ii)] If an edge $uv \in E$ has both $x_u, x_v > \frac{1}{2}$. Then $x^+_u, x^-_u, x^+_v, x^-_v \ge \frac{1}{2}$. Hence

          $$x^+_u + x^+_v \ge 1 \text{ and } x^-_u + x^-_v \ge 1.$$

          \begin{center}
            \begin{tikzpicture}
              % Draw the main line segment
              \draw[thick, -] (0,0) -- (8,0);

              % Mark the key points
              \draw[] (0,0) node[above=3pt] {$0$};
              \draw[] (4,0) node[above=3pt] {$\frac{1}{2}$};
              \draw[] (8,0) node[above=3pt] {$1$};

              % Add tick marks
              \draw (0,0.1) -- (0,-0.1);
              \draw (4,0.1) -- (4,-0.1);
              \draw (8,0.1) -- (8,-0.1);

              % Original points x_u and x_v
              \draw[fill=gustave] (5,0) circle (2pt) node[above=3pt] {$x_u$};
              \draw[fill=red] (6.4,0) circle (2pt) node[above=3pt] {$x_v$};

              % x^+ points (red)
              \draw[fill=gustave] (4.5,0) circle (2pt) node[above=3pt] {$x^+_u$};
              \draw[fill=red] (5.9,0) circle (2pt) node[above=3pt] {$x^+_v$};

              % x^- points (blue)
              \draw[fill=gustave] (5.5,0) circle (2pt) node[above=3pt] {$x^-_u$};
              \draw[fill=red] (6.9,0) circle (2pt) node[above=3pt] {$x^-_v$};
            \end{tikzpicture}
          \end{center}


  \end{itemize}

  \break

  Since there is an optimal solution to (LP) that is also an extreme point, we conclude there exists an optimal solution $x^*$ such that $x^*_v \in \{0, \frac{1}{2}, 1\}$ for every $v \in V$.

  Define the set $S_1$ of vertices with value 1 in $x^*$ and similarly the sets $S_0$ and $S_{1/2}$.

  \begin{center}
    \begin{tikzpicture}
      % Box for S_1
      \node at (1,3.3) {$S_0$};
      \fill[green, opacity=0.2, rounded corners] (0,0) rectangle (2,3);

      % Box for S_0
      \node at (3,3.3) {$S_1$};
      \fill[gustave, opacity=0.2, rounded corners] (2,0) rectangle (4,3);
      % Box for S_{1/2}
      \node at (5,3.3) {$S_{1/2}$};
      \fill[gray, opacity=0.2, rounded corners] (4,0) rectangle (6,3);

      \segment{0.75}{0.25}{2.75}{0.25}
      \segment{3}{0.75}{3}{2.75}
      \segment{3.25}{0.25}{5.25}{0.25}
      \segment{5}{0.75}{5}{2.75}
    \end{tikzpicture}
  \end{center}
  Possible cases of edges in $E$ are shown above.
\end{frame}

\begin{frame}{Solutions to (IP) from (LP)}
  \begin{mytheorem}[Nemhauser-Trotter]
    Let $x^*$ be an optimal solution to (LP).  Then there exists an optimal solution of (IP) that generates a vertex cover $S\subset V$ such that $S_1 \subset S \subset S_1 \cup S_{1/2}$.
  \end{mytheorem}
\end{frame}

\begin{frame}{Vertex Cover from (LP)}
  \textit{Proof.} Firstly, we show that $S\subset S_1 \cup S_{1/2}$. Suppose not i.e. $C_0 = S\cap S_0$ is not empty.

  For every vertex $v \in C_0$, there can only be edges between $v$ and vertices in $S_1$.
  \begin{center}
    \begin{tikzpicture}
      % Box for S_1
      \node at (1,3.3) {$S_0$};
      \fill[green, opacity=0.2, rounded corners] (0,0) rectangle (2,3);

      % Box for S_0
      \node at (3,3.3) {$S_1$};
      \fill[gustave, opacity=0.2, rounded corners] (2,0) rectangle (4,3);
      % Box for S_{1/2}
      \node at (5,3.3) {$S_{1/2}$};
      \fill[gray, opacity=0.2, rounded corners] (4,0) rectangle (6,3);

      % Box for S (vertex cover) - covers bottom half
      \draw[thick, red, rounded corners] (0,0) rectangle (6,1.5);
      \node at (6.2,0.75) {$S$};
    \end{tikzpicture}
  \end{center}
\end{frame}

\begin{frame}[allowframebreaks]{Vertex Cover from (LP)}
  Edges from $C_0$ to $\overline{C}_1=S_1\setminus S$ are covered once by $S$.
  \begin{center}
    \begin{tikzpicture}
      % Box for S_1
      \node at (1,3.3) {$S_0$};
      \fill[green, opacity=0.2, rounded corners] (0,0) rectangle (2,3);

      % Box for S_0
      \node at (3,3.3) {$S_1$};
      \fill[gustave, opacity=0.2, rounded corners] (2,0) rectangle (4,3);
      % Box for S_{1/2}
      \node at (5,3.3) {$S_{1/2}$};
      \fill[gray, opacity=0.2, rounded corners] (4,0) rectangle (6,3);

      % Box for S (vertex cover) - covers bottom half
      \draw[thick, red, rounded corners] (0,0) rectangle (6,1.5);
      \node at (6.2,0.75) {$S$};

      % Edge from C_0 to \overline{C}_1
      \segment{1}{0.75}{3}{2.25}
    \end{tikzpicture}
  \end{center}
  \break
  If $w(\overline{C}_1) \le w(C_0)$, we can get a better solution by choosing $\overline{C}_1$ instead of $C_0$.
  \begin{center}
    \begin{tikzpicture}
      % Box for S_1
      \node at (1,3.3) {$S_0$};
      \fill[green, opacity=0.2, rounded corners] (0,0) rectangle (2,3);

      % Box for S_0
      \node at (3,3.3) {$S_1$};
      \fill[gustave, opacity=0.2, rounded corners] (2,0) rectangle (4,3);
      % Box for S_{1/2}
      \node at (5,3.3) {$S_{1/2}$};
      \fill[gray, opacity=0.2, rounded corners] (4,0) rectangle (6,3);

      % Box for S (vertex cover) - covers S_1 and bottom half of S_{1/2}
      \draw[thick, red, rounded corners] (2,0) -- (2,3) -- (4,3) -- (4,1.5) -- (6,1.5) -- (6,0) -- cycle;
      \node at (6.2,0.75) {$S$};

      % Edge from C_0 to \overline{C}_1
      \segment{1}{0.75}{3}{2.25}
    \end{tikzpicture}
  \end{center}

  If $w(C_0) < w(\overline{C}_1)$, we use the perturbation technique again: add a small $\epsilon$ to every vertex in $C_0$ and subtract $\epsilon$ from every vertex in $\overline{C}_1$ to get a better feasible solution to (LP), contradicting the optimality of $x^*$.

  \break

  Now we prove that $S_1 \subset S$. Suppose not, i.e. $\overline{C}_1$ is nonempty. Let $v \in \overline{C}_1$.

  \begin{center}
    \begin{tikzpicture}
      % Box for S_1
      \node at (1,3.3) {$S_0$};
      \fill[green, opacity=0.2, rounded corners] (0,0) rectangle (2,3);

      % Box for S_0
      \node at (3,3.3) {$S_1$};
      \fill[gustave, opacity=0.2, rounded corners] (2,0) rectangle (4,3);
      % Box for S_{1/2}
      \node at (5,3.3) {$S_{1/2}$};
      \fill[gray, opacity=0.2, rounded corners] (4,0) rectangle (6,3);

      % Box for S (vertex cover) - covers S_1 and bottom half of S_{1/2}
      \draw[thick, red, rounded corners] (2,0) -- (2,2.5) -- (4,2.5) -- (4,1.5) -- (6,1.5) -- (6,0) -- cycle;
      \node at (6.2,0.75) {$S$};

      \fill[gustave] (3,2.75) circle (2pt) node[right=2pt] {$v$};
    \end{tikzpicture}
  \end{center}

  If $w(v) = 0$, include $v$ in $S$ anyway (the value of (IP) does not change).

  Consider the case $w(v) > 0$. Note that $v$ cannot have neighbors in $S_0$. Hence, we can decrease $x_v$ from 1 to $\frac{1}{2}$ and get a better feasible solution to (LP), contradicting the optimality of $x^*$.
\end{frame}



